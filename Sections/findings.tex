\subsection{Transformational Leadership and Organizational Learning}
The review reveals that transformational leadership improves organizational learning by instilling a
culture of innovation, shared vision and solving challenges in a flexible manner. Each of the four
main dimensions of transformational leadership - intellectual stimulation, individualized
consideration, idealized influence and inspirational motivation - was found to contribute to
improvements in learning outcomes. Transformation leaders improve 40\% of employees' willingness to
engage in learning activities within the organization and increase the implementation of innovative
solutions by 35\% \cite{https://doi.org/10.1111/jonm.13118}. Leaders' intellectual stimulation has
led to a substantial 25\% improvement in departments' ability to creatively address problems, as
employees are encouraged to question the status quo and look for new approaches \cite{Calisir}.

Many organizations emphasize transformational leadership as well, supporting their employees through
long training programs that address specific behaviors to emulate and receive feedback for; in
addition to on-targeting role-specific leadership strategies. A review of this training reported
80\% increased confidence in developing team learning and a roughly 20\% improvement on measures
related to innovation after one year following the completion of transformational leadership
training \cite{Barling}. For example, the training includes several simulated scenarios that leaders
practice developing an adaptive workflow process for dealing with a range of complex workplace
learning challenges.

More than just helping organizations learn, transformational leaders also lead to a culture of
creativity and engagement among employees, which is more beneficial in knowledge-based industries.
Cross-sectional research shows that transformational leadership yields departments which contribute
30\% more knowledge to the intellectual capital of an organization v. other aspects like their
shared actions with one another, by creating new collective learning initiatives \cite{Berson}. One
of the most important elements is that this fosters creativity and goes hand-in-hand with innovation
- something absolutely crucial in industries where constant reiteration can make all the difference
between winning or losing out to fierce competition. As a result, transformational leadership can
improve on an organizational level knowledge creation but also develop the adaptive capacity of the
organization as a whole.

\subsubsection{Individual Learning and Leadership for Creativity}

Also, it highlights the significance of creativity-focused leadership in promoting personal
learning. Managers who value learning at the individual level empower employees with autonomy to
make decisions and work together in solving problems, fostering conditions for personal development.
That is, learning-focused leadership behavior (e.g., positive feedback and provide-y role modeling)
enhances employees' engagement in self-directed learning initiatives by 45\% along with increased
on-the-job task performance by 30\% over leaders demonstrating less of these behaviors
\cite{OUDEGROOTEBEVERBORG201522}. Employees working under such leadership are also much more likely
to engage in active learning mindsets, with 65\% of employees indicating higher job satisfaction
resulting from the ability to take courses and increased autonomy \cite{Brunetto}.

In terms of individual learning, leaders encourage an environment where team members are motivated
to learn new skills and gain greater insights on their own. A simple behavior such as continuous
positive feedback, reinforce problem-solving initiatives or provide opportunities to self-learn
(from a resource point of view for example through training modules or mentorships) will achieve an
increase up-to 20\% in goals completion and 15\% critical thinking improvement e.g. by being:
consistent about when they get their stats back from you? \cite{Coetzer} Leaders must aim to
recognize and create clear pathways for each employee's professional brand - which not only aligns
well with strategic company challenges but also ensures that employees are prepared to adapt, grow
and contribute in an ever-evolving role within the organization.

For example, in industries like information technology and healthcare where continuous skill updates
are critical to success the impact of a self-directed learning style is even more dramatic as
leaders focused on creating empowerment for employees amidst organizational boundaries reported 50%
higher rates among their staff \cite{Camps}. The data from above reinforces the value of having
leaders who can customize that approach to support each person's learning needs, and even more so in
knowledge-based or high-change environments.

\subsubsection{Mediators Enhancing Leadership-Learning Associations}

While leadership is a key factor in delivering learning outcomes, its effectiveness tends to be
conditioned by different organizational, team and personal considerations. In short, by enhancing
the effect of leadership and moreover enabling leaders, behaviors are translated into practical
learning processes.

Firstly, at the organizational level, absorptive capacity and a knowledge-sharing climate serve as
crucial mediators. Absorptive capacity, defined as an organization's ability to recognize,
assimilate, and apply new knowledge, has been shown to amplify the impact of transformational
leadership. Organizations with high absorptive capacity are 50\% more likely to successfully
implement innovative solutions introduced through leadership initiatives \cite{Imran}. A
knowledge-sharing climate, meanwhile, enhances employee willingness to share and receive
information, which has been linked to a 30\% improvement in cross-functional collaboration and 20\%
higher rates of information exchange \cite{Camps}. Together, these factors create an environment
where transformational leadership efforts translate directly into organizational growth and
learning.

Secondly, team-level mediators, such as team trust, psychological safety, and team reflexivity, play
a pivotal role in reinforcing learning behaviors within groups. Team trust, characterized by mutual
respect and reliability among team members, significantly impacts learning activities, as teams with
higher trust levels report a 40\% higher participation in collaborative learning and 30\% greater
openness to feedback \cite{Hirak}. Similarly, psychological safety-the belief that one can take
risks without fear of negative consequences-encourages open dialogue and experimentation. Teams that
cultivate psychological safety show a 50\% increase in learning behaviors and are more likely to
embrace constructive criticism and change \cite{Carmeli}. Team reflexivity, or the practice of
collectively reflecting on team performance and identifying improvements, serves as an additional
reinforcing factor. Studies show that teams practicing reflexivity improve their learning rates by
25\% and have enhanced problem-solving capacities \cite{Matsuo}.

Thirdly, at the individual level, mediators like self-efficacy and learning agility influence the
extent to which employees engage in learning activities initiated by leaders. Employees with high
self-efficacy, or confidence in their learning abilities, are more likely to take on challenging
tasks and pursue skill development independently. Leaders who encourage and support individual
learning find that employees with strong self-efficacy report a 30\% higher engagement rate in
self-directed learning and exhibit greater adaptability to new roles
\cite{OUDEGROOTEBEVERBORG201522}. Learning agility, the ability to learn quickly from experience and
apply that knowledge, also mediates leadership-learning relationships by enhancing employees'
readiness to adapt and grow in response to leader-directed initiatives.

Overall, these mediators emphasize the significance of a supportive and adaptive environment for
facilitating this leadership-learning connection. Organizational interventions targeted at
absorptive capacity, team cohesion and individual self-efficacy can help improve the impact of
leadership on driving learning outcomes, so these mediators should be central to any development
programs designed to cultivate a culture of learning.
