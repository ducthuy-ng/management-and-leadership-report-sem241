\subsection{Comparing with slides and lessons}
Based on what the main points of the paper that our team has just described, we have done some
re-evaluation, and linked with what we learned from class sessions. Some of the points that we would
like to point out would be as follows.

First, we did agree that, to cultivate for learning at any different levels, from individual up to
organizational, it is necessary that a matching culture must be set up. In an environment where
innovation is encouraged, development can be built upon. One of the case studies that our team can
recall is "Google Creates Unique Culture", in which Google, to outcompete its competitor in the
Search engine market, relies very much based on its employees innovations and cutting-edge
technology. This is also true for neo-charismatic leadership. To be a neo-charismatic leader is to
be visionary. Similar to Rachel Adams and her Independent Center for Clinical Research (ICCR),
having a visionary leader helps motivate with the learning and the growth of the research center,
resulted in many successes at the early time.

In our class lessons, leadership styles are discussed in broader terms, encompassing a range of
approaches that include trait, behavioral, and situational theories. This text offers a
comprehensive overview of different models and their applications, from traditional leadership
theories to contemporary perspectives. While Leadership and Learning at Work zeroes in on the
concept of learning-oriented leadership, normal textbooks takes a wider lens, providing a
foundational understanding of how various styles can support or hinder learning. The concept of
transformational leadership, which motivates and empowers team members, serves as a bridge between
the texts, as both recognize its potential to create a supportive learning environment.

Finally, Leadership and Learning at Work presents learning-oriented leadership as a distinct,
integrative approach that combines coaching, feedback, and collaboration to sustain organizational
development. Meanwhile, textbooks and slides explore similar ideas but does not prioritize learning
as the central focus of leadership. Instead, it explores a variety of goals that leadership can
serve. Overall, while both texts acknowledge the importance of growth and adaptability, Leadership
and Learning at Work treats learning as a key objective, whereas the slides situate it as one of
many possible outcomes within the broader landscape of leadership effectiveness.

\subsection{Comparing with textbook - "Leadership Theory \& Practice"}
As a textbook for this subject, \textit{Leadership Theory \& Practice} is what we would also like to
take into comparison when researching about the topic of growing team members. Both of these
documents examine the role of leadership in supporting organizational growth and development, but
they differ significantly in their focal points. \textit{Leadership Theory \& Practice} provides a
broad overview of leadership theories, including transformational, transactional, servant, and
situational leadership, and discusses how each theory can be applied in diverse organizational
contexts. It emphasizes how different leadership approaches impact organizational success and
employee motivation. In contrast, Leadership and Learning at Work narrows its focus specifically to
learning-oriented leadership, exploring how leadership styles and behaviors directly foster learning
and knowledge-sharing within the workplace.

In terms of practical application, \textit{Leadership Theory \& Practice} presents leadership as a
flexible tool that can be adapted to various organizational goals, such as improving productivity or
innovation. While learning and development are part of this framework, they are not the central
objective. On the other hand, Leadership and Learning at Work frames learning as the core function
of effective leadership, arguing that fostering a continuous learning environment is crucial for
organizational adaptability and resilience. It introduces the concept of learning-oriented
leadership as a model that prioritizes skills development and knowledge transfer, positioning
leaders as coaches who facilitate structured reflection and collaborative learning.

Another distinction lies in how each source treats the outcomes of leadership. \textit{Leadership
Theory \& Practice} explores a broad range of outcomes, such as team cohesion, performance
improvement, and employee satisfaction. It sees learning as one of many paths toward these ends.
Leadership and Learning at Work, however, sees learning as a primary outcome of leadership and
positions it as essential for long-term success. By promoting specific strategies for creating a
learning-centered culture, such as feedback loops and reflective practices, it redefines leadership
effectiveness in terms of how well leaders can cultivate an adaptable, knowledge-driven workforce.
